
\section{Operational Transformation}

\subsection{Introduction} 
Consistency maintenance is significant challenge in any collaborative ediotor.
Most innovative technique for maintaing it is Operational Transformation.
We concentarte on integrative review of all OPT techniques and 
their issues,algorithms and achievements\cite{1998}

\subsection{Preliminaries}
In this section some basic concepts,definitions and terminologies are introduced.
Operational transformation technique allows two type of operations called local operations and remote opeartions.
Local operations are nothing but operations that are genearted by local site.
Where as remote opeartions are those operations taht are generated at some other site other than local site.
Local opeartions are executed immediately with out any changes but remote opeartions have to be transformed against existing operations before execution.  
\\Now we introduce some defnitions.

{\bf Def 1}:causal ordering relation \cite{1998}$$ \rightarrow $$
\\Given two operations Oa and Ob,generated at sites i and j,then $$Oa \rightarrow Ob$$,if (1)i=j and generation of Oa happened before
the generation of Ob or (2)i is not equal to j and execution of  Oa at site j happened before the generation of Ob or (3)there exists an
opeartion Ox,such that $$Oa \rightarrow Ox$$ and $$Ox \rightarrow Ob$$.

{\bf Def 2}:Dependent and Independent operations
\\Given two operations Oa and Ob,(1)Ob is dependent on Oa if $$Oa \rightarrow Ob$$.(2)Oa and Ob are independent(or concurrent),
expressed as $$Oa \parallel Ob$$,if neither $$Oa \rightarrow Ob$$ nor $$Ob \rightarrow Oa$$


\subsection{Principles of consistency}

\begin{itemize}
  


 \item convergence(all copies are identical)
 \item Casuality preservation
 \item  Intention preservation(effect of execution of operation O is same as intention of O)

\end{itemize}


\subsection{List of algorithms}\cite{1998}

\begin{itemize}
 \item GROVE
 \item REDUCE
 \item JUPITER
 \item  adOPTed

\end{itemize}
\subsubsection{GROVE approach}
\begin{itemize}
  \item {
    Replicated architecture(for good responsiveness)
  }
  \item {
    Locally executed and then braodcasted
  }
  \item {
   solution consists of 2 components
   \begin{itemize}
   \item {
   state-vector time stamping scheme for precedence property
   }
   \item{
   dOPT(distributed) algorithm for convergence
   }
   \end{itemize}
  }
  
  
  \item {
    dOPT algorithm requires the transformation function to satisfy the following property
    \begin{itemize}
    \item {
    $$O_a \circ O_b' = O_b \circ O_a'$$
    }
    \end{itemize}
  }
   \item {
   dOPT algorithm 
  }
  \item {
   unsolved dOPT puzzle
  }
  \item {
  Reason:above TP has to be applied only when two operations are in same state 
  }
  \item{
  If they are not in same state? dOPT did not handle that !!
  }

  \end{itemize}
  
  
  
  
  \subsubsection{REDUCE approach}
  \begin{itemize}
  \item {
  Fully didtributed and Replicated as in GROVE  
  }
  \item {
    Linear history buffer is maintained instead of log
  }
  \item {
   Also handles 3rd consistency principle(Intention Preservation)
  }
  \item {
  To address that it uses IT/ET and pre/post conditions for applying those
  }
  \item{
  specifications
  }
  \item{
 GOT(generic operational transformation algorithm)
  }
  \end{itemize}

  
  
  
\subsubsection{JUPITER approach}
  \begin{itemize}
  \item {
    No broadcasting among clients
  }
  \item {
    only 2-way between client and server
  }
 
  \end{itemize}


  
  
\subsubsection{adOPTed approach}
  \begin{itemize}
  \item {
    It requires additional properties from GROVE for transformation function to satisfy
    \begin{itemize}
    \item{
     $$O_a \circ O_b' = O_b \circ O_a'$$
    } 
    \item{
   $$T(T(O,O_a),O_b')=T(T(O,O_b),O_a')$$ 
    }
    
    
    \end{itemize}
  }
  \item {
    TP1 ensures unique vertex labelling 
  }
  \item {
   TP2 ensures unique edge labelling
  }
 
  \end{itemize}
  
  
  
  
  
  
  
  
  
  
  
  
  
  
  
  
  \subsubsection{treeOPT algorithm}\cite{2003}
  
    one of the keys to customization is structured representation of the document.
  
    In this section we summarize treeOPT algorithm.
 
   It applies operational transformation mechanism recursively over different levels of document.
  
  Advantages of using this algorithm are
  \begin{itemize}
  \item better efficiency
  
  \item possibility of working at different granularity levels
  
 \item improvement in semantic consistency
 \end{itemize}
 











   This algorithm  uses tree representation of the document.
  
    Disadvantages of linear representation of document are
    \begin{itemize}
    \item {
    keeps single history of operations
    }
    \item{ 
    Less efficient as whole history needs to be scanned for every operation(Though they are working on different sections of the document)
    }
    \end{itemize}








  




%The report is organized as follows: Section 1 provides the background knowledge and the need for a continuous authentication model for smart phones.
%Section 2 discusses about general experiment setup.
%Section 3 discusses about the various approaches and their performance in authentication. Future scope is discussed in section 5. 
%Section 6 draws the conclusions for the work done.